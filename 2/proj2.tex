\documentclass[11pt,a4paper,twocolumn]{article}
%packages
\usepackage{amsmath}
\usepackage{amsthm}
\usepackage{amssymb}
\usepackage[czech]{babel}
\usepackage[utf8]{inputenc}
%\usepackage[T1]{fontenc}
\usepackage[left=1.5cm,text={18cm, 25cm},top=2.5cm]{geometry}
\providecommand{\uv}[1]{\quotedblbase #1\textquotedblleft}


%settings
\renewcommand*\rmdefault{ptm}

%definitions
\theoremstyle{definition}
\newtheorem{definice}{Definice}[section]

\theoremstyle{plain}
\newtheorem{algoritmus}[definice]{Algoritmus}

\theoremstyle{plain}
\newtheorem{veta}{Věta}

%document
\begin{document}
\begin{titlepage}
\begin{center}
\Huge
\textsc{Fakulta informačních technologií\\
Vysoké učení technické v~Brně\\}
\vspace{\stretch{0.382}}
\LARGE
Typografie a publikování\,--\,2. projekt \\
Sazba dokumentů a matematických výrazů
\vspace{\stretch{0.618}}
\end{center} 
{\Large 2016 \hfill
Jan Koscielniak}
\end{titlepage}
\section*{Úvod}
 
V~této úloze si vyzkoušíme sazbu titulní strany, matematických vzorců, prostředí a dalších textových struktur obvyklých pro technicky zaměřené texty, například rovnice \eqref{rovnice1} nebo definice \ref{def1} na straně \pageref{def1}.

Na titulní straně je využito sázení nadpisu podle optického středu s~využitím zlatého řezu. Tento postup byl probírán na přednášce.

\section{Matematický text}
Nejprve se podíváme na sázení matematických symbolů a výrazů v~plynulém textu. Pro množinu $ V $ označuje $ \textrm{card}(V) $ kardinalitu $ V $.
Pro množinu $ V $ reprezentuje $ V^{*} $ volný monoid generovaný množinou $ V $ s~operací konkatenace.
Prvek identity ve volném monoidu $ V^{*} $ značíme symbolem $ \varepsilon $.
Nechť $ V^{+} = V^{*} - \lbrace\varepsilon\rbrace $. Algebraicky je tedy  $ V^{+} $ volná pologrupa generovaná množinou $ V $ s~operací konkatenace.
Konečnou neprázdnou množinu $ V $ nazvěme \textit{abeceda}.
Pro $ w \in V^{*} $ označuje $ \vert w\vert $ délku řetězce $ w $. Pro $ W \subseteq V $ označuje $ \textrm{occur}(w,W) $ počet výskytů symbolů z~$ W $ v~řetězci $ w $ a $ \textrm{sym}(w,i) $ určuje $ i $-tý symbol řetězce $ w $; například $ \textrm{sym}(abcd,3) = c $.

Nyní zkusíme sazbu definic a vět s~využitím balíku \texttt{amsthm}.
\begin{definice}\label{def1}
\textit{Bezkontextová gramatika} je čtveřice $ G = (V,T,P,S) $, kde $ V $ je totální abeceda,
$ T \subseteq V $ je abeceda terminálů, $ S \in (V-T) $ je startující symbol a $ P $ je konečná množina pravidel
tvaru $ q\colon A \rightarrow \alpha $, kde $ A \in (V-T)$, $ \alpha \in V^{*} $ a $ q $ je návěští tohoto pravidla. Nechť $ N = (V - T) $ značí abecedu neterminálů.
Pokud $ q\colon A \rightarrow \alpha \in P $ ,$ \gamma,\delta \in V^{*} $, $ G $ provádí derivační krok z~$ \gamma A\delta $ do $ \gamma\alpha\delta $ podle pravidla $ q\colon A \rightarrow \alpha $, symbolicky píšeme 
$ \gamma A\delta \Rightarrow \gamma\alpha\delta \  [ q\colon A \rightarrow \alpha ]$ nebo zjednodušeně $ \gamma A\delta \Rightarrow \gamma\alpha\delta $. Standardním způsobem definujeme $ \Rightarrow^{m} $, kde $ m\geq 0 $. Dále definujeme 
tranzitivní uzávěr $ \Rightarrow^{+} $ a tranzitivně-reflexivní uzávěr $ \Rightarrow^{*} $.
\end{definice}


Algoritmus můžeme uvádět podobně jako definice textově, nebo využít pseudokódu vysázeného ve vhodném prostředí (například \texttt{algorithm2e}).
\begin{algoritmus}\label{alg1}
Algoritmus pro ověření bezkontextovosti gramatiky. Mějme gramatiku $ G = (N,T,P,S) $.
\begin{enumerate}
 \item\label{step1}Pro každé pravidlo $ p \in P $ proveď test, zda $ p $ na levé straně obsahuje právě jeden symbol z~$ N $.
 \item Pokud všechna pravidla splňují podmínku z~kroku \ref{step1}, tak je gramatika $ G $ bezkontextová.
\end{enumerate}
\end{algoritmus}

\begin{definice}
Jazyk definovaný gramatikou $ G $ definujeme jako $ L(G) = \lbrace w \in T^{*}\vert S \Rightarrow^{*} w \rbrace $.
\end{definice}
\subsection{Podsekce obsahující větu}
\begin{definice}
Nechť $ L $ je libovolný jazyk. $ L $ je \textit{bezkontextový jazyk}, když a jen když $ L = L(G) $, kde $ G $ je libovolná bezkontextová gramatika.
\end{definice}
\begin{definice}
Množinu $ \mathcal{L}_{CF} = \lbrace L\vert L $ je bezkontextový jazyk $ \rbrace $ nazýváme třídou \textit{bezkontextových jazyků}.
\end{definice}
\begin{veta}\label{veta1}
Nechť $ L_{abc} = \lbrace a^{n}b^{n}c^{n}\vert n\geq 0 \rbrace $. Platí, že $ L_{abc} \notin \mathcal{L}_{CF} $.
\end{veta}
\begin{proof}[Důkaz]
Důkaz se provede pomocí Pumping lemma pro bezkontextové jazyky, kdy ukážeme, že není možné, aby platilo, což bude implikovat pravdivost věty \ref{veta1}.
\end{proof}

\section{Rovnice a odkazy}

Složitější matematické formulace sázíme mimo plynulý text. Lze umístit několik výrazů na jeden řádek, ale pak je třeba tyto vhodně oddělit, například příkazem \verb;\quad;. 
\begin{equation*}{0}
\sqrt[x^{2}]{y^{3}_{0}} \quad \mathbb{N} = \lbrace 0,1,2,\ldots\rbrace \quad x^{y^{y}} \neq x^{yy} \quad z_{i_{j}} \not\equiv z_{ij} 
\end{equation*}

V~rovnici (\ref{rovnice1}) jsou využity tři typy závorek s~různou explicitně definovanou velikostí.
\begin{eqnarray}\label{rovnice1}
\bigg\lbrace \Big[ \big(a+b\big)*c\Big]^{d} +1\bigg\rbrace &=& x  \\
\lim_{x\to\infty}\frac{\sin^{2}{x} + \cos^{2}{x}}{4} &=& y \nonumber
\end{eqnarray}
V~této větě vidíme, jak vypadá implicitní vysázení limity $ \lim_{n\to\infty} f(n) $ v~normálním odstavci textu. Podobně je to i s~dalšími symboly jako $ \sum^{n}_{1} $ či $ \bigcup_{A\in\mathcal{B}} $. V~případě vzorce  $ \lim\limits_{x\to 0} \dfrac{\sin{x}}{x} = 1 $ jsme si vynutili méně úspornou sazbu příkazem \verb;\limits;.

\begin{eqnarray}\label{rovnice2}
\int\limits_{a}^{b} f(x)\,\textrm{d}x &=& - \int_{b}^{a} f(x)\,\textrm{d}x  \\
\Big(\sqrt[5]{x^{4}}\Big) ^{\prime} = \Big(x^{\frac{4}{5}}\Big)^{\prime} &=& \frac{4}{5}x^{-\frac{1}{5}} = \dfrac{4}{5\sqrt[5]{x}} \\
\overline{\overline{A\vee B}} &=& \overline{\overline{A}\wedge\overline{B}}
\end{eqnarray}

\section{Matice}

Pro sázení matic se velmi často používá prostředí \texttt{array} a závorky (\verb;\left;, \verb;\right;). 

\begin{align*}
&\left(\begin{array}{cc}
a+b & b-a \\
\widehat{\xi + \omega} & \hat{\pi} \\
\vec{a} & \overleftrightarrow{AC}\\
0 & \beta
\end{array}\right)\\
\mathbf{A} = &\left\Vert\begin{array}{cccc}
a_{11} & a_{12} & \ldots & a_{1n} \\
a_{21} & a_{22} & \ldots & a_{2n} \\
\vdots & \vdots & \ddots & \vdots \\
a_{m1} & a_{m2} & \ldots & a_{mn} \\
\end{array}\right\Vert\\
&\left\vert\begin{array}{cc}
t & u\\
v~& w \\
\end{array}\right\vert
= tw- uv
\end{align*}
Prostředí \texttt{array} lze úspěšně využít i jinde.
\begin{equation*}
\dbinom{n}{k}  = \Bigg\lbrace
\begin{array}{ll}
\frac{n!}{k!(n-k)!} & \textrm{pro } 0 \leq k~\leq n \\
0 & \textrm{pro } k~< 0 \textrm{ nebo } k~> n
\end{array}
\end{equation*}


\section{Závěrem}

V~případě, že budete potřebovat vyjádřit matematickou konstrukci nebo symbol a nebude se Vám dařit jej nalézt v~samotném \LaTeX u, doporučuji prostudovat možnosti balíku maker \AmS -\LaTeX.
\end{document}
