\documentclass[11pt,a4paper]{article}
%packages
\usepackage[czech]{babel}
\usepackage[utf8]{inputenc}
\usepackage[left=2cm,text={17cm, 24cm},top=3cm]{geometry}
\providecommand{\uv}[1]{\quotedblbase #1\textquotedblleft}
\usepackage{natbib} 
%settings
\renewcommand*\rmdefault{ptm}

%document
\begin{document}
\begin{titlepage}
\begin{center}
\Huge
\textsc{Vysoké učení technické v~Brně}\\
\huge
\textsc{Fakulta informačních technologií\\}
\vspace{\stretch{0.382}}
\LARGE
Typografie a publikování\,--\,4. projekt \\
\Huge
Citace
\vspace{\stretch{0.618}}
\end{center} 
{\Large 13. dubna 2016 \hfill
Jan Koscielniak}
\end{titlepage}
\section*{O Typografii}

\bibliography{literatura}
Mnohé techonologie využívaly principů umělé inteligence již před více než půl stoletím.
První práce na tomto poli vyšla roku 1943.
Jejími autory byli Warren McCulloch a Walter Pitts~\cite{NegnevitskyMichael2002Ai}. 
S nárůstem výkonu výpočetní techniky však došlo v posledních letech k obroskému rozmachu v tomto oboru~\citep{What_is_artificial_intelligence}.
Na umělou intelgenci se dá nahlížet z mnoha pohledů.
S pomocí biologie, matematiky, logiky a různých dalších věd usiluje o napadobení kongnitivního procesu~\cite{RagasLudek2017Ui}. 
Na podobné bazí pracuje několik dalších přístupů, jako například genetické programování, které z poznatků umělé inteligence čerpají~\cite{FITPUB11427}.
Umělá inteligence nachází využití v širokém spektru oblastí.
Jedním z nejfrekventovanějších způsobů uplatnění umělé inteligence je v současné době oblast tvorby her.
Například Jan Černohub ve své bakalářské prácí využil principů umělé inteligence k určení nejvhodnějších tahů ve hře Carcassonne~\cite{CernohubJan2010Uipd}.
Tím jak se umělá inteligence zdokonaluje, začíná se uvažovat o prosazení tohoto oboru také v oblastech, kde to dříve nebylo myslitelné.
Touto oblastí je například soudnictví, kde má pomoci vyhodnocovat efektivitu a správnost soudních řízení~\cite{PahAdamR2022TPoA}.
Důležitou roli může také hrát ve zeefektivňění rekvalifikací zaměstatnanců~\cite{RobsonRobby2022IlAc}.


S rozvojem umělé inteligence se však stále více akcentuje problematika jejích možných nebezpečí.
Většína badatelů se shoduje, že nebezpečí hrozí především ve dvou hlavních směrech.
Zaprvé, pokud je umělá inteligence naprogramována za účelem ničení. 
Nebezpečná však může být také, pokud je naprogramována k pozitivnímu účelu, avšak k jeho vykonání zvolí destruktivní metody~\cite{Future}.
Rizika spojená s rychlým rozvojem tohoto odvětví si uvědomuje taktéž široká veřejnost.
Podle výzkumu společnosti Tech Pro Research vyjadřuje v tomto ohledu obavy až 34 procent respodentů~\cite{Maddox_2015}.
Co však bude, až umělá inteligence dosáhne tak vysokého stupně rozvoje, že dalece předčí tu lidskou?
Na tuto otázku se snaží odpovědět Max Tegmark, jenž se touto problematinkou zabývá dlouhodobě~\cite{TegmarkMax2020Z3:c}.


\newpage
\renewcommand{\refname}{Literatura}
\bibliographystyle{csplainnat}

\end{document}
