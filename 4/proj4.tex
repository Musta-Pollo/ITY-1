\documentclass[11pt,a4paper]{article}
%packages
\usepackage[czech]{babel}
\usepackage[utf8]{inputenc}
\usepackage[left=2cm,text={17cm, 24cm},top=3cm]{geometry}
\providecommand{\uv}[1]{\quotedblbase #1\textquotedblleft}
\usepackage{natbib} 
%settings
\renewcommand*\rmdefault{ptm}

%document
\begin{document}
\begin{titlepage}
\begin{center}
\Huge
\textsc{Vysoké učení technické v~Brně}\\
\huge
\textsc{Fakulta informačních technologií\\}
\vspace{\stretch{0.382}}
\LARGE
Typografie a publikování\,--\,4. projekt \\
\Huge
Citace
\vspace{\stretch{0.618}}
\end{center} 
{\Large 13. dubna 2016 \hfill
Jan Koscielniak}
\end{titlepage}
\section*{O Typografii}

Vyvážit dokument tak, aby obsahoval hodnotné informace a přitom byl i příjemný na pohled není nic jednoduchého. Jednou z~hlavních zásad je: \uv{\emph{Používejte maximálně dva nebo tři druhy písma}} \citep{samara:2008}. Přemírou zdobného písma, užitého ve snaze zkrášlit dokument lze dosáhnout pravého opaku. Zdobných písem sice vzniká mnoho, ale málokteré lze užít dokumentu ku prospěchu \citep{Beran:2012}. Typografie by měla podporovat pointu textu a nikoliv ji přebíjet. \citep{Butterick:2010}

A~jsou to právě nezdobná písma, která se používají desetiletí. Například slavný Times New Roman, který poprvé spatřil světlo světa 3.10.1932 \citep{Tholenaar:2010}. Toto tvrzení dokládají i slova českého grafika Zdeňka Zieglera, který říká, že se stejně po čase od nových písem vrací k~lety osvědčeným písmům \citep{Krc:2012}. 

Mnoho typografů se nejdřívě věnovalo jiné umělecké profesi a \uv{náhodou} se dostali k~typografii, z~těch českých například Josef Váchal nebo Oldřich Hlavsa \citep{Storm:2008}. Lze samozřejmě najít i příklady, které předchozímu tvrzení odporujují. Je jím například slavný typograf Eric Gill, který věnoval typografii celý svůj život \citep{Gill:2014}. 

Typografie nachází uplatnění i mimo dokumenty nebo knihy. S~nástupem internetu vznikla i nutnost užívání typografie na webových stránkách. Ty se ve svých začátcích zaměřovaly spíše na obsah a kód, než na vzhled. Pouze několik větších firem tehdy mělo typograficky korektní a na pohled pěkný design. Těmito a dalšími problémy z~oblasti typografie se zabýval časopis Deleatur až do roku 2003 \citep{Deleatur:2012}. Častým problémem webových stránek, zejména podnikových je špatné definování parametrů písma \citep{Cecetka:2010}. Další hrubou chybou je nezachování typografické konzistence na celém webu \citep{Kyrnin:2016}. Naopak dobrou praktikou je zvolení primárního a sekundárního fontu \citep{Johnson:2015}.

Typografie je často opomíjená i ve státní správě, a i proto vznikla diplomová práce Marty Gawin, která navrhla systém vizuální identity pro polskou státní správu viz \citep{Gawin:2011}. Nabízí se zde srovnání s~českou státní správou, která opakovaně selhává ve snaze unifikovat vizuální styl, kterým by se důstojně prezentovala \citep{Pecina:2012}. 
\newpage
\renewcommand{\refname}{Literatura}
\bibliographystyle{csplainnat}
\bibliography{literatura}

\end{document}
